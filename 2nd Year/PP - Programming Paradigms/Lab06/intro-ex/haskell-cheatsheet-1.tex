%%%%%%%%%%%%%%%%%%%%%%%%%%%%%%%%%%%%%%%%%%%%%%%%%%%%%%%
% MatPlotLib and Random Cheat Sheet
%
% Edited by Michelle Cristina de Sousa Baltazar
%
% http://matplotlib.org/api/pyplot_summary.html
% http://matplotlib.org/users/pyplot_tutorial.html
%
%%%%%%%%%%%%%%%%%%%%%%%%%%%%%%%%%%%%%%%%%%%%%%%%%%%%%%%

\documentclass[a4paper]{article}
\usepackage[landscape]{geometry}
\usepackage{url}
\usepackage{multicol}
\usepackage{amsmath}
\usepackage{amsfonts}
\usepackage{tikz}
\usetikzlibrary{decorations.pathmorphing}
\usepackage{amsmath,amssymb}
\usepackage{hyperref}

\usepackage{colortbl}
\usepackage{xcolor}
\usepackage{mathtools}
\usepackage{amsmath,amssymb}
\usepackage{enumitem}

\usepackage{textcomp} 
\usepackage{couriers}
\usepackage{listings}
\lstset{
	numbers			= left,
	numberstyle		= \tiny,
    numbersep       = 5pt,
	captionpos		= b,
	breaklines		= true,
	basicstyle		= \ttfamily\footnotesize, 
	tabsize			= 4,
	escapeinside	= {~}{~},
}
\lstdefinelanguage{Racket}{
  morekeywords=[1]{define, define-syntax, define-macro, lambda, define-stream, stream-lambda},
  morekeywords=[2]{begin, call-with-current-continuation, call/cc,
    call-with-input-file, call-with-output-file, case, cond,
    do, else, for-each, if,
    let*, let, let-syntax, letrec, letrec-syntax,
    let-values, let*-values,
    and, or, not, delay, force,
    quasiquote, quote, unquote, unquote-splicing,
    map, fold, syntax, syntax-rules, eval, environment },
  morekeywords=[3]{import, export},
  alsodigit=!\$\%&*+-./:<=>?@^_~,
  sensitive=true,
  morecomment=[l]{;},
  morecomment=[s]{\#|}{|\#},
  morestring=[b]",
  basicstyle=\footnotesize\ttfamily,
  keywordstyle=\color[rgb]{0,.3,.7},
  commentstyle=\color[rgb]{0.133,0.545,0.133},
  stringstyle={\color[rgb]{0.75,0.49,0.07}},
  upquote=true,
  breaklines=true,
  breakatwhitespace=true,
  literate=*{`}{{`}}{1}
}

\title{Haskell - Intro}
\usepackage[brazilian]{babel}
\usepackage[utf8]{inputenc}

\advance\topmargin-1.0in
\advance\textheight3in
\advance\textwidth3in
\advance\oddsidemargin-1.5in
\advance\evensidemargin-1.5in
\parindent0pt
\parskip1pt
\newcommand{\hr}{\centerline{\rule{3.5in}{1pt}}}
%\colorbox[HTML]{e4e4e4}{\makebox[\textwidth-2\fboxsep][l]{texto}
\begin{document}

\begin{center}{\huge{\textbf{Haskell CheatSheet}}}\\
{\large Laborator 6}
\end{center}

\begin{multicols*}{3}

\tikzstyle{mybox} = [draw=black, fill=white, very thick,
    rectangle, rounded corners, inner sep=10pt, inner ysep=10pt]
\tikzstyle{fancytitle} =[fill=black, text=white, font=\bfseries]

% Mihnea
\tikzstyle{mybox_code} = [mybox, draw = orange, fill=sandybrown]
\tikzstyle{fancytitle_code} = [fancytitle, fill = orange]

\definecolor{almond}{rgb}{0.94, 0.87, 0.8}
\definecolor{apricot}{rgb}{0.98, 0.81, 0.69}
\definecolor{atomictangerine}{rgb}{1.0, 0.6, 0.4}
\definecolor{sandybrown}{rgb}{0.96, 0.64, 0.38}
\definecolor{buff}{rgb}{0.94, 0.86, 0.51}

\definecolor{persianred}{rgb}{0.8, 0.2, 0.2}
\definecolor{papayawhip}{rgb}{1.0, 0.94, 0.84}
\tikzstyle{mybox_persianred} = [mybox, draw = persianred, fill=papayawhip]
\tikzstyle{fancytitle_persianred} = [fancytitle, fill = persianred]

\definecolor{whitesmoke}{rgb}{0.96, 0.96, 0.96}
\definecolor{wenge}{rgb}{0.39, 0.33, 0.32}
\tikzstyle{mybox_blue} = [mybox, draw = wenge, fill=whitesmoke]
\tikzstyle{fancytitle_blue} = [fancytitle, fill = wenge]

\definecolor{cerise}{rgb}{0.87, 0.19, 0.39}
\definecolor{mistyrose}{rgb}{1.0, 0.89, 0.88}
\tikzstyle{mybox_cerise} = [mybox, draw = cerise, fill=mistyrose]
\tikzstyle{fancytitle_cerise} = [fancytitle, fill = cerise]

\definecolor{pinegreen}{rgb}{0.0, 0.47, 0.44}
\definecolor{bubbles}{rgb}{0.91, 1.0, 1.0}
\tikzstyle{mybox_pinegreen} = [mybox, draw = pinegreen, fill=bubbles]
\tikzstyle{fancytitle_pinegreen} = [fancytitle, fill = pinegreen]

\definecolor{cream}{rgb}{1.0, 0.99, 0.82}
\definecolor{mikadoyellow}{rgb}{1.0, 0.77, 0.05}
\tikzstyle{mybox_mikadoyellow} = [mybox, draw = mikadoyellow, fill=cream]
\tikzstyle{fancytitle_mikadoyellow} = [fancytitle, fill = mikadoyellow]

\definecolor{cornsilk}{rgb}{1.0, 0.97, 0.86}
\tikzstyle{mybox_orange} = [mybox, draw = orange, fill=cornsilk]
\tikzstyle{fancytitle_orange} = [fancytitle, fill = orange]

\definecolor{aliceblue}{rgb}{0.94, 0.97, 1.0}
\definecolor{seagreen}{rgb}{0.18, 0.55, 0.34}
\tikzstyle{mybox_seagreen} = [mybox, draw = seagreen, fill=aliceblue]
\tikzstyle{fancytitle_seagreen} = [fancytitle, fill = seagreen]

\definecolor{jazzberryjam}{rgb}{0.65, 0.04, 0.37}
\definecolor{almond}{rgb}{0.94, 0.87, 0.8}
\tikzstyle{mybox_jazzberryjam} = [mybox, draw = jazzberryjam, fill=almond]
\tikzstyle{fancytitle_jazzberryjam} = [fancytitle, fill = jazzberryjam]

\definecolor{amaranth}{rgb}{0.9, 0.17, 0.31}
\definecolor{bisque}{rgb}{1.0, 0.89, 0.77}
\tikzstyle{mybox_amaranth} = [mybox, draw = amaranth, fill=bisque]
\tikzstyle{fancytitle_amaranth} = [fancytitle, fill = amaranth]

\definecolor{carminered}{rgb}{1.0, 0.0, 0.22}
\definecolor{blanchedalmond}{rgb}{1.0, 0.92, 0.8}
\tikzstyle{mybox_carminered} = [mybox, draw = amaranth, fill=blanchedalmond]
\tikzstyle{fancytitle_carminered} = [fancytitle, fill = carminered]

\definecolor{midnightgreen}{rgb}{0.0, 0.29, 0.33}
\definecolor{lavendermist}{rgb}{0.9, 0.9, 0.98}
\tikzstyle{mybox_midnightgreen} = [mybox, draw = midnightgreen, fill=lavendermist]
\tikzstyle{fancytitle_midnightgreen} = [fancytitle, fill = midnightgreen]

\definecolor{indigo}{rgb}{0.29, 0.0, 0.51}
\definecolor{isabelline}{rgb}{0.96, 0.94, 0.93}
\tikzstyle{mybox_indigo} = [mybox, draw = indigo, fill=isabelline]
\tikzstyle{fancytitle_indigo} = [fancytitle, fill = indigo]

\definecolor{russet}{rgb}{0.5, 0.27, 0.11}
\definecolor{ivory}{rgb}{1.0, 1.0, 0.94}
\tikzstyle{mybox_russet} = [mybox, draw = russet, fill=ivory]
\tikzstyle{fancytitle_russet} = [fancytitle, fill = russet]

\definecolor{neongreen}{rgb}{0.12, 0.58, 0.02}
\definecolor{splashedwhite}{rgb}{0.9, 0.99, 0.9}
\tikzstyle{mybox_neongreen} = [mybox, draw = neongreen, fill=splashedwhite]
\tikzstyle{fancytitle_neongreen} = [fancytitle, fill = neongreen]

%---------------------------------------------------------------------------------

\begin{tikzpicture}
\node [mybox_persianred] (box){%
    \begin{minipage}{0.3\textwidth}

\begin{lstlisting}[language=Haskell, numbers=none]
5       :: Int
'H'     :: Char
"Hello" :: String
True    :: Bool
False   :: Bool
\end{lstlisting}
	\end{minipage}
};

\node[fancytitle_persianred, right=10pt] at (box.north west) {Tipuri de bază};
\end{tikzpicture}

\begin{tikzpicture}
\node [mybox_seagreen] (box){%
    \begin{minipage}{0.3\textwidth}
    	{\centering\bf\small\color{seagreen} :t \\}

\begin{lstlisting}[language=Haskell, numbers=none]
> :t 42            
42 :: Num a => a
\end{lstlisting}

\textbf{a} reprezintă o variabilă de tip, 
restrictionată la toate tipurile numerice.

\begin{lstlisting}[language=Haskell, numbers=none]
> :t 42.0
42 :: Fractional a => a
\end{lstlisting}

In acest exemplu, \textbf{a} este
restrictionată la toate tipurile numerice fracționare
(e.g. \textbf{Float}, \textbf{Double}).

	\end{minipage}
};

\node[fancytitle_seagreen, right=10pt] at (box.north west) {Determinarea tipului unei expresii};
\end{tikzpicture}

\begin{tikzpicture}
\node [mybox_jazzberryjam] (box){%
    \begin{minipage}{0.3\textwidth}
    	{\centering\bf\small\color{jazzberryjam} [] (:) \\}

\begin{lstlisting}[language=Haskell, numbers=none]
[]              -- lista vida
(:)             -- operatorul de adaugare 
                -- la inceputul listei

1 : 3 : 5 : []  -- lista care contine 1, 3, 5
[1, 3, 5]       -- sintaxa echivalenta
\end{lstlisting}

	\end{minipage}
};

\node[fancytitle_jazzberryjam, right=10pt] at (box.north west) {Constructori liste};
\end{tikzpicture}

\begin{tikzpicture}
\node [mybox_mikadoyellow] (box){%
    \begin{minipage}{0.3\textwidth}
{\centering\bf\small\color{mikadoyellow} not  \&\&   $||$  \\}
		\begin{lstlisting}[language=Haskell, numbers=none]
not True                      False
not False                     True
True || False                 True
True && False                 False
        \end{lstlisting}
    \end{minipage}
};

\node[fancytitle_mikadoyellow, right=10pt] at (box.north west) {Operatori logici};
\end{tikzpicture}

\begin{tikzpicture}
\node [mybox_orange] (box){%
    \begin{minipage}{0.3\textwidth}
    	{\centering\bf\small\color{orange} (++) head tail last init take drop\\}
		\begin{lstlisting}[language=Haskell, numbers=none]
[1, 2] ++ [3, 4]                     [1, 2, 3, 4]

head [1, 2, 3, 4]                    1
tail [1, 2, 3, 4]                    [2, 3, 4]

last [1, 2, 3, 4]                    4
init [1, 2, 3, 4]                    [1, 2, 3]

take 2 [1, 2, 3, 4]                  [1, 2]
take 2 "HelloWorld"                  "He"

drop 2 [1, 2, 3, 4]                  [3, 4]

null []                              True
null [1, 2, 3]                       False
\end{lstlisting}
    \end{minipage}
};

\node[fancytitle_orange, right=10pt] at (box.north west) {Operatori pe liste};
\end{tikzpicture}

\begin{tikzpicture}
\node [mybox_blue] (box){%
    \begin{minipage}{0.3\textwidth}
	{\centering\bf\small\color{wenge} length elem reverse\\}
		\begin{lstlisting}[language=Haskell, numbers=none]
length [1, 2, 3, 4]                  4

elem 3 [1, 2, 3, 4]                  True
elem 5 [1, 2, 3, 4]                  False

reverse [1, 2, 3, 4]                 [4, 3, 2, 1]
        \end{lstlisting}
    \end{minipage}
};

\node[fancytitle_blue, right=10pt] at (box.north west) {Alte operații};
\end{tikzpicture}

%---------------------------------------------------------------------------------

\begin{tikzpicture}
\node [mybox_amaranth] (box){%
    \begin{minipage}{0.3\textwidth}
% {\centering\bf\small\color{midnightgreen}    Tupluri}

Spre deosebire de liste, tuplurile au un număr fix de elemente, iar acestea
pot avea tipuri diferite.
\begin{lstlisting}[language=Haskell, numbers=none]
import Data.Tuple

("Hello", True) :: (String, Bool)
(1, 2, 3)       :: (Integer, Integer, Integer)

fst ("Hello", True)     "Hello"
snd ("Hello", True)     True
swap ("Hello", True)    (True, "Hello") 
\end{lstlisting}

    \end{minipage}
};

\node[fancytitle_amaranth, right=10pt] at (box.north west) {Tupluri};
\end{tikzpicture}


\begin{tikzpicture}
\node [mybox_midnightgreen] (box){%
    \begin{minipage}{0.3\textwidth}
{\centering \bf\small\color{midnightgreen} \textbackslash arg1 arg2 $\rightarrow$ corp\\}

\begin{lstlisting}[language=Haskell, numbers=none]
\x -> x                       functia identitate
(\x y -> x + y) 1 2           3
let f = \x y -> x + y         legare la un nume
(f 1 2)                       3
\end{lstlisting}
\end{minipage}
};

\node[fancytitle_midnightgreen, right=10pt] at (box.north west) {Funcții anonime (lambda)};
\end{tikzpicture}

%---------------------------------------------------------------------------------

\begin{tikzpicture}
\node [mybox_indigo] (box){%
    \begin{minipage}{0.3\textwidth}

\begin{lstlisting}[language=Haskell, numbers=none]
-- if .. then .. else
factorial x = 
    if x < 1 then 1 else x * factorial (x - 1)

-- guards
factorial x
    | x < 1 = 1
    | otherwise = x * factorial (x - 1)

-- case .. of
factorial x = case x < 1 of
    True -> 1
    _    -> x * factorial (x - 1)

-- pattern matching
factorial 0 = 1
factorial x = x * factorial (x - 1)

\end{lstlisting}
\end{minipage}
};

\node[fancytitle_indigo, right=10pt] at (box.north west) {Definire functii};
\end{tikzpicture}

% %---------------------------------------------------------------------------------

\begin{tikzpicture}
\node [mybox_cerise] (box){
    \begin{minipage}{0.3\textwidth}
    
In Haskell funcțiile sunt, by default, in forma curry.
    
\begin{lstlisting}[language=Haskell, numbers=none]

:t (+) 
(+) :: Num a => a -> a -> a

:t (+ 1)
(+ 1) :: Num a => a -> a

\end{lstlisting}
\end{minipage}
};

\node[fancytitle_cerise, right=10pt] at (box.north west) {Curry};
\end{tikzpicture}

\begin{tikzpicture}
\node [mybox_neongreen] (box){
    \begin{minipage}{0.3\textwidth}
    {\centering\bf\small\color{neongreen} map filter foldl foldr zip zipWith\\}
\begin{lstlisting}[language=Haskell, numbers=none]

map     :: (a -> b) -> [a] -> [b]
filter  :: (a -> Bool) -> [a] -> [a]
foldl   :: (a -> b -> a) -> a -> [b] -> a
zip     :: [a] -> [b] -> [(a, b)]
zipWith :: (a -> b -> c) -> [a] -> [b] -> [c]


map (+ 2) [1, 2, 3]          [3, 4, 5]

filter odd [1, 2, 3, 4]      [1, 3]

foldl (+) 0 [1, 2, 3, 4]     10
foldl (-) 0 [1, 2]           -3   (0 - 1) - 2
foldr (-) 0 [1, 2]           -1   1 - (2 - 0)

zip [1, 2] [3, 4]            [(1, 3), (2, 4)]
zipWith (+) [1, 2] [3, 4]    [4, 6]

\end{lstlisting}
\end{minipage}
};

\node[fancytitle_neongreen, right=10pt] at (box.north west) {Funcționale uzuale};
\end{tikzpicture}
%---------------------------------------------------------------------------------

\end{multicols*}
\end{document}
Contact GitHub API Training Shop Blog About
© 2016 GitHub, Inc. Terms Privacy Security Status Help